\documentclass{article}
\usepackage[utf8]{inputenc}

\title{An Android Bitcoin Wallet Application}
\author{Benjamin Russell Moore}
\date{11-09-2018}

\begin{document}

\maketitle

I would like to build a Bitcoin wallet application for Android smartphones. This
application would provide a safe way for Android users to interface with the
Bitcoin blockchain while providing them with information that they need in order
to make a purchase using bitcoin. This would include providing live price data
of bitcoin so they can convert into their chosen fiat currency in order to make
sure they're not over or underpaying for products as well as providing a level
of security so that even if someone else were to use their phone they wouldn't
be able to make purchases without the users authorisation.

\vspace{5mm}

Keywords: Mobile Application, Blockchain, Bitcoin

\section{To Produce}

A mobile application that allows users to send and receive Bitcoin while also
providing live Bitcoin price information to the user to advise them in paying
the correct amount of Bitcoin for any given purchases.

\section{Method of Approach}

\begin{center}
\begin{tabular}{ l l }
 Agile & Software development process \\ 
 Kotlin with Android SDK & For developing the front-end application \\  
 Bitcoind & To provide a JSON-RPC interface to the Bitcoin blockchain \\  
 Go & For developing the back-end API wrapper for Bitcoind \\  
 Git and Github & Development source control and hosting
\end{tabular}
\end{center}

\section{Learning Requirements}

\begin{itemize}

\item Learn how to provide a secure connection to the Bitcoin blockchain from
    my android application using the Bitcoin JSON-RPC

\item Learn Go to write efficient back end code

\item Improve on my current knowledge of Android development by learning
    Kotlin in order to build a modern Android application codebase 

\end{itemize}

\section{Potential Risks and Courses of Action}

\begin{itemize}

\item Loss of project codebase - I am going to be using Github for repository
  hosting. By commiting to the repository on a regular basis I will have a
  stream of backups I can revert to if there is an issue. 

\item Hardware failure - In the event of a hardware failure I will move
  development to my laptop or utilise the university labs in order to complete
  my project while using the android simulator as a test-bed.

\item Difficulty meeting all of my learning requirements - If I have difficulty
  meeting my learning requirements, I will have to more intently focus my
  research on functionality of the application I'm struggling the most with.

\item Congestion in the Blockchain slows the development process - I will be
  using the Bitcoin test-net for the development of my application which is
  purely designed for the testing and debugging of blockchain interfacing
  applications. In the event that blockchain becomes congested the test-net
  should remain fast enough to develop with as it contains no real money and
  just consists of developers reading and writing dummy transactions

\end{itemize}

\end{document}